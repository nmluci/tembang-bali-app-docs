\documentclass{screenplay}[2012/06/30]
\usepackage[utf8]{inputenc}

\title{Trailer Tembang Bali App}
\author{deema}
\address{Jl. Cokroaminoto. No 214 \\
Denpsasar \\ 80116 \\
085155334059 }

\begin{document}
\maketitle

\intslug[]{Pura}
Diawali dengan shoot pura, dan tempat persembahyangan, diiringi dengan lagu gamelan. Ceritanya ada acara pasraman materi tentang dharma gita.
\begin{dialogue}{Person 1}
    Eh, kok kamu baru dateng?, abis darimana?
\end{dialogue}
\begin{dialogue}{Person 2}
    Ni baru dateng dari latihan nok langsung
\end{dialogue}
\begin{dialogue}{Person 1}
    Udah bawa buku dharma gita?, kan disuruh kemarin materinya sekarang itu.
\end{dialogue}
\begin{dialogue}{Person 2}
    Kle lupa nok, baru dateng dari latihan langsung kesini aku. Gimana men ni dimarah ntar ama bu loli?
\end{dialogue}
Orang ketiga masuk mendekat.
\begin{dialogue}{Person 3}
    Kenapa-kenapa? ni
\end{dialogue}
\begin{dialogue}{Person 2}
    Ini si "orang 1" ga bawa buku dharma gita
\end{dialogue}
\begin{dialogue}{Person 3}
    Oh, ini bisa pake, aku ada buat aplikasi, namanya "Tembang Bali", bisa kalian pake untuk belajar
\end{dialogue}
\begin{dialogue}{Person 3}
    Fiturnya lengkap mulai dari ada sekar rare sampe sekar agung ada. Ini bakal di update juga terus karena database nya open source.
\end{dialogue}
\begin{dialogue}{Person 1}
    Oh gitu ya?
\end{dialogue}
\begin{dialogue}{Person 3}
    Iyaa, lengkap bisa dipake untuk belajar juga ada fitur karaokenya. Ada juga sejarah singkat soal setiap gita nya
\end{dialogue}
\begin{dialogue}{Person 1}
    Oh bagus berarti ni, kasian juga liat kakekku bukunya udh rusak, bisa juga berarti buat persiapan belajar tugas basa bali di sekolah
\end{dialogue}
\begin{dialogue}{Person 3}
   Iyaa, download nake sekarang. Aplikasinya bisa didownload langsung di link dibawah, nanti akan segera datang ke playstore juga.
\end{dialogue}
\extslug[]{Pura}

\end{document}