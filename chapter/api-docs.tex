
\section{Dokumentasi PocketBase}
Pada iterasi awal ini, Tembang Bali menyimpan seluruh data dalam 3 buah tabel, yaitu \textit{songs}, \textit{song\_types}, dan \textit{song\_subtypes}.

\subsection*{Song Subtypes}
\textbf{GET} /api/collections/song\_subtypes/records?page=1\&perPage=1

\textbf{search}:

\indent mendapatkan seluruh subtype yang tersedia pada aplikasi

\textit{Response}:
\begin{lstlisting}[language=json,firstnumber=1]
{
    "page": 1,
    "perPage": 30,
    "totalItems": 5,
    "totalPages": 1,
    "items": [
        {
        "collectionId": "wpn5u2qwdejiy6d",
        "collectionName": "song_subtypes",
        "created": "2024-07-03 06:49:40.472Z",
        "id": "wdsyso7rsy259tg",
        "name": "Kidung Dewa Yadnya",
        "song_type": "1uyaj240u3jfxfm",
        "updated": "2024-07-03 06:49:40.472Z"
        }

    ]
}
        \end{lstlisting}

\textbf{view}:

\textbf{GET} /api/collections/song\_subtypes/records/:id

\indent mendapatkan seluruh subtype  yang tersedia pada aplikasi

\textit{Response}:
\begin{lstlisting}[language=json,firstnumber=1]
{
    "id": "RECORD_ID",
    "collectionId": "wpn5u2qwdejiy6d",
    "collectionName": "song_subtypes",
    "created": "2022-01-01 01:00:00.123Z",
    "updated": "2022-01-01 23:59:59.456Z",
    "song_type": "RELATION_RECORD_ID",
    "name": "test"
}
\end{lstlisting}

\subsection*{Song Types}
\textbf{GET} /api/collections/song\_types/records?page=1\&perPage=1

\textbf{search}:

\indent mendapatkan seluruh type yang tersedia pada aplikasi

\textit{Response}:
\begin{lstlisting}[language=json,firstnumber=1]
{
    "page": 1,
    "perPage": 30,
    "totalItems": 5,
    "totalPages": 1,
    "items": [
        {
            "id": "RECORD_ID",
            "collectionId": "3dz70i7dtob7rem",
            "collectionName": "song_types",
            "created": "2022-01-01 01:00:00.123Z",
            "updated": "2022-01-01 23:59:59.456Z",
            "name": "test",
            "description": "test"
        }
    ]
}

\end{lstlisting}

\textbf{view}:

\textbf{GET} /api/collections/song\_types/records/:id

\indent mendapatkan seluruh type yang tersedia pada aplikasi

\textit{Response}:
\begin{lstlisting}[language=json,firstnumber=1]
{
    "id": "RECORD_ID",
    "collectionId": "3dz70i7dtob7rem",
    "collectionName": "song_types",
    "created": "2022-01-01 01:00:00.123Z",
    "updated": "2022-01-01 23:59:59.456Z",
    "name": "test",
    "description": "test"
}
\end{lstlisting}

\subsection*{Songs}
\textbf{GET} /api/collections/song/records?page=1\&perPage=1

\textbf{search}:

\indent mendapatkan seluruh kidung/lagu yang tersedia pada aplikasi

\textit{Response}:
\begin{lstlisting}[language=json,firstnumber=1]
{
    "page": 1,
    "perPage": 30,
    "totalItems": 5,
    "totalPages": 1,
    "items": [
        {
            "id": "RECORD_ID",
            "collectionId": "aqly0n339u404q8",
            "collectionName": "songs",
            "created": "2022-01-01 01:00:00.123Z",
            "updated": "2022-01-01 23:59:59.456Z",
            "song_type": "RELATION_RECORD_ID",
            "song_subtype": "RELATION_RECORD_ID",
            "title": "test",
            "history": "test",
            "music_data": "filename.jpg",
            "music_additonal": "filename.jpg",
            "lyrics": "JSON",
            "lyric_string": "test",
            "image": "filename.jpg"
        }    
    ]
}
\end{lstlisting}

\textbf{view}:

\textbf{GET} /api/collections/song/records/:id

\indent mendapatkan seluruh kidung/lagu yang tersedia pada aplikasi

\textit{Response}:
\begin{lstlisting}[language=json,firstnumber=1]
{
    "id": "RECORD_ID",
    "collectionId": "aqly0n339u404q8",
    "collectionName": "songs",
    "created": "2022-01-01 01:00:00.123Z",
    "updated": "2022-01-01 23:59:59.456Z",
    "song_type": "RELATION_RECORD_ID",
    "song_subtype": "RELATION_RECORD_ID",
    "title": "test",
    "history": "test",
    "music_data": "filename.jpg",
    "music_additonal": "filename.jpg",
    "lyrics": "JSON",
    "lyric_string": "test",
    "image": "filename.jpg"
}
\end{lstlisting}